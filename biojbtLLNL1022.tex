\documentclass{article}


\title{}
\author{} 
\date{}


\begin{document}
\underline{\bf Biography:} Vivek Kale is an employee at a new company called Charmworks, Inc to develop Charm++ for commercial use. During his PhD, Vivek worked under the supervision of Professor William D. Gropp and was a Lawrence Scholar at Lawrence Livermore National Laboratory, working with mentors Todd Gamblin and Bronis de Supinski. Vivek's dissertation focused on developing efficient
loop scheduling strategies to improve the scalability of
bulk-synchronous MPI+OpenMP applications run on clusters of SMPs. He
also was involved in work on MPI shared memory extensions model for
MPI-3, also known as the MPI+MPI model. After completing his PhD, he
worked under Bill Gropp at XPACC as a postdoc to improve
combustion of regular mesh application codes for a year. 
After that, Vivek Kale was a Computer Scientist at the University
of Southern California's Information Sciences Institute in
Computational Science and Technology division, where he developed new
methods for MPI+CUDA application codes to do use user-defined
schedules and worked on combined load balancing and loop scheduling,
receiving a best poster nomination at SC17. Since the end of this past
June, Vivek has joined Charmworks, Inc. actively taking on projects
and being involved in making his ideas usable for consumers of HPC
applications, in particular develop libraries for loop scheduling and
tasking and their integration into the Charm++ Runtime System. Vivek continues to remain active in the OpenMP Language Committee, hosting the 2020 OpenMP forum at USC/ISI in Los Angeles.

%Vivek works on improving performance of scientific codes run on
%supercomputers with multi-core nodes. During his PhD, Vivek worked
%under the supervision of Professor William D. Gropp and was a Lawrence
%Scholar at Lawrence Livermore National Laboratory, working 
 
% with Todd Gamblin and Bronis de Supinski. Vivek's dissertation work
% focused on developing lightweight loop scheduling strategies to
% improve the scalability of bulk-synchronous MPI+OpenMP applications
% run on clusters of SMPs. He also was involved in work on MPI shared
% memory extensions model for MPI-3, also known as the MPI+MPI
% model. Vivek's current work involves performance optimizations that
% include techniques of auto-tuning and loop scheduling of
% ptychography solvers intended to run on supercomputers having nodes
% with GPUs and/or MICs.

% Vivek Kale did a PhD under Professor William D. Gropp with mentors
% Todd Gamblin and Bronis de Supinski. 

% 
%Vivek Kale worked at University of Southern California/ISI after that
%in the Computational Science and Technology group. 



\end{document}
